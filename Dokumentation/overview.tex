On page~\pageref{chap:unify} the task description contains a drafted solution called \textsc{SQTL} which is similar to \textsc{SQTL}. This
solution, even if still in work, may not be fast enough to run on small 8 Bit core nodes due to computing power. Additionally, a low-power 
setup does currently not provide enough space for complex query deciphering, at least not in the year 2012. Further development on a 
multifunctional language is the path for the future but a practical solution is not in the reach of \textsc{SQTL}.

Our approach is more direct and does not enable queries. This logic should be done on the end device, e.g. a Android-powered Smartphone or a 
vanilla i86 machine. Still, more complex commands are possible but require additional programming logic on the node's side. We try to 
keep the low-level efforts on the nodes as small as possible.

\cite{Heinzelman00energy-efficientcommunication} describes a system for a homogenous hardware setup so we skip this \textsc{Mit} development as 
our aim is not dependent on specific brands or models of wireless sensor nodes.

\section{Security in Node Communication}

Data sent through wireless nodes may contain protecable data. Note that nodes are used with military and civilan applications. As military does
not release open information about their encryption the civil development concentrates on envirnoment measuring as temperature, pressure and other 
non-important data and health monitoring.\cite{Dispert}
While health data output may not be very interesting to attackers the input side of specific critical health functions.\footnote{Imagine somebody 
just turns off an artificial cardiac clock.}
Our first plans included a security layer based on \textsc{PKCS\#11}. Diagram~\ref{XXX} contains a rough setup. \cite{UBISEMINAR} tried to 
use symmetric encryption based on \textsc{AES} which resulted in a 80\% increase in power consumption. A further option was described by 
using the hardware encryption on the Wireless Communication chip on-board but this solution is very specific to specific hardware.

For encryption a public-private key setup could result in a fine-grained setup but there is still no development on the civil side.
The solution we had planned was based on a \(I^2C\) connectable reader for micro-sized smartcards. This would allow a setup where every node
would be addressable by its public key. A sophisticated setup would be possible.\footnote{The reasons for stopping our efforts is on human side. Pretty
bad but not egliable to mention it in this paper.}


-connect various language sources
-different design board
-different generations
