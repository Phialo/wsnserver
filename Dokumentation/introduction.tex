\section{Hello World}
\subsection{HelloW.NET}
\subsubsection{XML HelloW.NET}

Wireless Sensor Nodes \footnote{We will refer instead of the long form to the abbreviation \textsc{WSN} in the remaining documentation.}
are versatile low-power embedded systems used for monitoring sensors, measuring concentration of various chemicals, meshed communication
and a multitude of other purposes.

They were developed by the DARPA[FIXME: CITE] in the year XXX. The main points which stopped the usage in civilian areas were:

\begin{itemize}
\item battery longlivety
\item size
\item portability
\item secure communication
\item calculation speed
\item compatibility
\item lacking unified language for high-level programming
\end{itemize}

If one \textsc{WSN} is not enough for a usage scenario some form of communication between the devices is neccessary. The obvious
route is through means of wireless communication.

Different forms of wireless communication are available:

\begin{description}
\item[ZigBee] Low power but limited range through the band XXX
\item[IEEE 802.11] Popular name is \textsc{WLAN}. Mainly the 2.4 GHz band is used but in subsets like 802.11a the short-waved 5GHz band is usable.
Energy consumption is generally higher.\footnote{Regulatory standards in Germany forbid more than 150mA total power sent.}
\item[Custom radio communication] XXX Ask Dispert
\item[More] XXX Ask Koss
\end{description}

Wireless sensor nodes can be deployed and used in a multitude of means.\footnote{But not for special cases as~\cite{biederbeck}}
