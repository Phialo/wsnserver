Different communication languages exist for the big communication layers Bluetooth, ZigBee and active RFID.

In~\cite[p. 407]{Akyildiz02wirelesssensor} a hardware query language for sensors called \textsc{SQTL}\footnote{Sensor Query and Tasking Language}
is decribed. This language is, very similar to current \textsc{SQL}, supposed to give a subset of nodes querys as \textit{Which sensor does receive
something irregular(Tempererature above some treshold or detection of dangerous chemicals in environment protection} and the nodes affected 
will return their data if they match the query. Other possiblities are also described in the aforementioned article but they go into a technical
dimension as reconfiguring the possible positions of nodes and general network configuration.

Such tasks are generally meant to be solved by a versatile language. As the possibilites are just drafed and still discussed in a comitee our 
approach is a rapid iterative development cycle supported with automated tests to develop a fast solution.

To overcome this problem and establish a user-defined set of generalized commands this project was started.
The original idea was to unify various communication level layers and their corresponding dialects to 
archive easy and centralized means of sending and receiving command, i.e. one typed command will result in 
a multitude of different bit-wise packages and vendor specific communication languages.
It is also suitable for managing heterogenous node deployment.\footnote{Imagine a system where \(\subset(language)=project teams_count * platform \) 
so either everybody uses some established language which does not exist(2012) or one project team is solely occupied with 
making a tranlator - for every project!}

The language from the nodes was a superset of a node project about distributed temperature measurement. This project was done by 
a fellow student which was not available throughout the development process. The commands he used are shown in~\ref{tab:nodecommands}.

\begin{table}[!h] 
\centering 
\begin{tabular}{|l||l|} 
Example Translation & Device Command \\ 
\hline
GetId & +WWSNID \\
GetChannel & +WCHAN \\
GetPanId & +WPANID \\
LedOn & +WLED1 ON \\
LedOff & +WLED1 OFF \\
TempOn & +WSENDTEMP \\
TempOff & +WNSENDTEMP \\
Enquiry & +WENQ \\
\end{tabular} 
\caption{ Table of Node Commands} 
\label{tab:nodecommands} 
\end{table}

The description is self-descriptive.\footnote{Note that internal programming requires \textit{linefeed} and \textit{carriage return}. We did not find out 
why but suppose Windows enthusiasts behind the original node project.}

Table contents were just chosen on decision of one members as they are customizable via a small dictionary in the \textsc{Python} programming language. We 
suppose even a non-expert in node programming would be able to suit it to his or her needs.\footnote{But we doubt he would use our libary anyway}
The left side of~\ref{tab:nodecommands} is the customizable one while the right side was already given. First three commands are about the 
custom localisation of one node while the next ones are self-explanational.\footnote{We still research what Enquiry is doing but the flying spaghetti monster is not giving us the needed inspiration}

[KAMIL ERKLAERT HIER DAS RESTFUL API]
